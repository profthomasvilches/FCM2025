\documentclass{abntex2}
\usepackage{graphicx} % Required for inserting images
\usepackage{float}
\usepackage[brazil]{babel}
\usepackage[utf8]{inputenc}
\usepackage{lipsum}
\usepackage{amsmath}


% Bibliografia

% Para bibliografia
%\RequirePackage{natbib}
%


\newcommand{\biblio}{%
%\vspace*{2.7cm}
\thispagestyle{empty}
\renewcommand\refname{Refer\^encias Bibliogr\'aficas}
\addcontentsline{toc}{section}{\MakeUppercase{Refer\^encias Bibliogr\'aficas}}
\bibliographystyle{vancouver}
\bibliography{\databib}
}
\newcommand{\databib}{bibliografia}
\renewcommand{\baselinestretch}{1.5}
\setlength{\textwidth}{16cm}


\title{Aula Ferramentas}
\author{Thomas N Vilches}
\date{Novembro 2025}

\pagenumbering{arabic}

\begin{document}

\maketitle


\section{Introdução}


\lipsum[1]


\subsection{Dengue}

\lipsum[1-6]

A Equação \ref{eq:segundograu}

\begin{equation}
    f(x) = \frac{x^2+2x+1}{2} \label{eq:segundograu}
\end{equation}


\begin{equation*}
    f(x) = \frac{x^2+2x+1}{2}
\end{equation*}


\subsubsection{Dengue subsection}

\subsection{Transmissão}

Note as melhores práticas de Word escritas na Figura \ref{fig:bestpratices} extraido de \cite{moghadas2021evaluation}.

\begin{figure}[H]
    \centering
    \includegraphics[width=0.15\linewidth]{89256656_2845969895488190_1798273966129807360_n.png}
    \caption{Melhores práticas Word. Retirado de \cite{moghadas2021evaluation}}
    \label{fig:bestpratices}
\end{figure}

\newpage

\lipsum[2]


\begin{table}
    \centering
    \begin{tabular}{l|cc}
        1 & 2 & 3\\
        \hline
        4 & 5 & 6\\
        7 & 8 & 9\\
    \end{tabular}
    \caption{Caption}
    \label{tab:placeholder}
\end{table}


\begin{table}[h]
    \centering
    \caption{Parâmetros do Modelo \cite{vilches2021}}
    \label{tab:parametros}
    \begin{tabular}{c l c l}
        \toprule
        \textbf{Parâmetro} & \textbf{Significado} & \textbf{Valor} & \textbf{Referência} \\
        \midrule
        $\mu_m$ & Taxa de mortalidade do mosquito & $1/3$ \text{weeks}$^{-1}$ & \text{[ref]} \\
        $\gamma$ & Inverso do período infeccioso em humanos & $1.0$ \text{weeks}$^{-1}$ & \text{[ref]} \\
        $\eta$ & Inverso do período latente em humanos & $1.4$ \text{weeks}$^{-1}$ & \text{[ref]} \\
        $\nu_m$ & Inverso do período latente em mosquitos & $1.0$ \text{weeks}$^{-1}$ & \text{[ref]} \\
        \bottomrule
    \end{tabular}
\end{table}


Referenciando \cite{vilches_disease_2019,ministerio_da_saude_estrategia_2024}

% \begin{thebibliography}{99}

% \bibitem{AutorAno}
% S. Autor, \textit{Título do Artigo ou Livro}. Nome da Editora ou Revista, ano.

% \bibitem{OutroAutorAno}
% F. Outro Autor, “Título do Trabalho”, \textit{Nome da Conferência ou Periódico}, vol. 10, pp. 20--35, 2020.

% \end{thebibliography}

\biblio


\end{document}
